\documentclass[9pt,a4paper,oldfontcommands,doubleside]{memoir} %%%Para A4


\usepackage[utf8]{inputenc}
\usepackage{embedfile}
\usepackage{xypic}
\usepackage{listings}%%% DOR
\usepackage[spanish]{babel}
\usepackage{framed,graphicx}
\usepackage{amsmath,amssymb,amsthm,verbatim}
\usepackage[sections=normal]{savetrees}
\usepackage[pdftex,pdfnewwindow=true]{hyperref}
\usepackage[dvipsnames*,svgnames]{xcolor}
\usepackage{mfpic}
\usepackage{mathrsfs}
\usepackage{dor}
%%\DeclareUnicodeCharacter{00A0}{~}
\newtheorem{Tma}{Teorema}


\hypersetup{
    colorlinks,
    linkcolor={red!50!black},
    filecolor={green!50!black},
    urlcolor={blue!80!black}
}


\setlength{\FrameSep}{4pt}
\setcounter{maxsecnumdepth}{3}

\newcommand\HRule{\noindent\rule{\linewidth}{1.5pt}}
\makeatletter
    \renewcommand\part{%
      \if@openright
        \cleardoublepage
      \else
        \clearpage
      \fi
      \thispagestyle{empty}%
      \if@twocolumn
        \onecolumn
        \@tempswatrue
      \else
        \@tempswafalse
      \fi
      \null\vfil
      \secdef\@part\@spart}
\makeatother

\usemetapost
\usemplabels

\newcommand{\destaca}[1]{{\color{red}\mathbf{#1}}}


\begin{document}
\frontmatter

\begin{titlingpage}
\vspace*{\stretch{3}}
\HRule
\begin{flushright}
{\LARGE \sffamily JRE RHG BLM DOR FQG}\\
 \hspace{1cm} \\
 {\sffamily \bfseries \Huge Laboratorio}
\end{flushright}
\HRule
\vspace*{\stretch{2}}
\large
\begin{center}
{\sc Madrid 2016}
\end{center}
\end{titlingpage}

\begin{KeepFromToc}
  \tableofcontents
\end{KeepFromToc}



\chapter*{Pr\'ologo}
\monta|prologo|
\mainmatter

\part{Fundamentos}

\chapter{Introducci\'on}
\thispagestyle{empty}
\monta|cap-1-interface|
%\section{Notas personales}
%\monta|notas-cap1|

\chapter{SAGE como calculadora avanzada}
\thispagestyle{empty}
\monta|cap-2-calculadora|
%\section{Notas personales}
%\montan|notas-cap2|

\chapter{Estructuras de datos}\label{estr-dat}
\thispagestyle{empty}
\monta|cap-3-estructuras_datos|
%\section{Notas personales}
%\montan|notas-cap3|

\chapter{T\'ecnicas de programaci\'on}\label{prog}
\thispagestyle{empty}
\monta|cap-4-programacion|
%\section{Notas personales}
%\montan|notas-cap4|

\chapter{Complementos}
\thispagestyle{empty}
\monta|cap-5-complementos|
%\section{Notas personales}
%\montan|notas-cap5|

\part{Aplicaciones}

\chapter{Teoría de números}\label{tn1}
\monta|cap-6-tnumeros1|
%%\section{Notas personales}
%%\montan|notas-cap6|


\chapter{Aproximaci\'on}\label{aprox}
\monta|cap-7-aproximacion|
%%\section{Notas personales}
%%\montan|notas-cap7|




\chapter{Criptograf\'{\i}a}\label{cript}
\monta|cap-8-cripto|
%\section{Notas personales}
%\montan|notas-cap8|

\chapter{M\'as teor\'{\i}a de n\'umeros}
\monta|cap-9-tnumeros2|
%\section{Notas personales}
%%\montan|notas-cap9|

\chapter{Matem\'atica discreta}\label{discr}
\monta|cap-10-discreta|
%\section{Notas personales}
%%\montan|notas-cap10|




\chapter{Probabilidad}\label{prob}
\monta|cap-11-probabilidad|
%\section{Notas personales}
%%\montan|notas-cap11|


\chapter{Miscel\'anea}
\monta|cap-12-miscelanea|
%\section{Notas personales}
%%\montan|notas-cap12|





\begin{appendices}
\chapter{Recursos}
\monta|bibliografia|
\end{appendices}
\end{document}

