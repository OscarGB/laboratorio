La {\itshape teor\'{\i}a de n\'umeros} o {\itshape aritm\'etica} estudia las
propiedades los n\'umeros enteros y  de algunos otros conjuntos de n\'umeros
que generalizan a los enteros. 

Una de las caracter\'{\i}sticas de la teor\'{\i}a de n\'umeros es la existencia
de una gran cantidad de {\sc problemas de enunciado muy sencillo\footnote{Para
muchos
de ellos basta con conocer {\itshape las cuatro reglas}.}
pero de difícil, o desconocida, demostración}.
En~todos estos casos se han hecho comprobaciones masivas mediante ordenador  sin
encontrar un contraejemplo, de forma que se cree que los resultados son ciertos
pero en muchos de ellos se cree tambi\'en que la demostraci\'on est\'a
todav\'{\i}a lejos.


Muchos de los ejercicios planteados en este cap\'{\i}tulo son
{\itshape ejercicios ret\'oricos}, es decir, son ejercicios de programaci\'on,
que como tales son \'utiles para aprender a programar, pero que no nos dicen
gran cosa sobre los problemas matem\'aticos, extremadamente dif\'{\i}ciles, que
estudian. 


En otros casos plantearemos algunos ejercicios cuya respuesta ya debe ser
conocida, te\'oricamente, a trav\'es de 
la asignatura {\itshape Conjuntos y
N\'umeros}, y por tanto son tambi\'en, en otro sentido, {\itshape ret\'oricos}.
En particular,  supondremos conocidos los siguientes contenidos:

\begin{enumerate}
 \item Las nociones de grupo abeliano, anillo y cuerpo. Las definiciones
b\'asicas acerca del anillo de los n\'umeros enteros y el cuerpo de los
racionales. El teorema de factorizaci\'on de enteros como producto de primos, a
veces llamado {\itshape teorema fundamental de la aritm\'etica.} 
 
 \item La definici\'on de los {\itshape anillos $\mathbb{Z}_m$ de clases de
restos m\'odulo un entero $m$} y el hecho de que son cuerpos \'unicamente
aquellos en los que
el m\'odulo es primo. 

\item Las nociones de {\itshape m\'aximo com\'un divisor (MCD)} y {\itshape
m\'{\i}nimo com\'un m\'ultiplo (MCM)}, el algoritmo de Euclides para calcular el
MCD y su consecuencia, {\itshape el teorema de Bézout}. Este \'ultimo resultado
lo usamos para calcular el inverso de un elemento invertible en un anillo de
clases de restos y, por tanto,  es importante.

\item El {\itshape teorema peque\~no de Fermat}, y su generalizaci\'on,
{\itshape el teorema de Fermat-Euler},  que estudian las potencias de elementos
en un anillo de clases de restos. En el enunciado del teorema de Fermat-Euler
aparece la funci\'on $\phi$ de Euler, que tambi\'en supondremos conocida.


Estos resultados ser\'an utilizados m\'as
adelante en criptograf\'{\i}a y al estudiar algunos criterios para determinar si
un entero es primo.
 
\end{enumerate}



\section{Clase de restos}

Fijamos un entero $m$ al que llamamos {\itshape m\'odulo}. 
En el anillo de los n\'umeros enteros la relaci\'on  $nR_{n}n^\prime 
$ si  y s\'olo si   $n$ y $n^\prime$ tienen el mismo resto al 
dividir por $m$ es una relaci\'on de equivalencia. Denotamos por $[ 
n]_m$ {\itshape la clase de $n$ m\'odulo $m$},  aunque habitualmente  escribimos
$[n]$
sin que el m\'odulo aparezca expl\'{\i}citamente en la notaci\'on.



El conjunto cociente tiene $m$ elementos que corresponden a los 
$m$ posibles restos, es decir, una clase,$[0]$, contiene todos 
aquellos enteros que al dividir entre $m$ dan resto $0$ (m\'ultiplos 
de $m$), otra, $[1]$, aquellos que dan resto $1$, y as\'{\i}  hasta $m-
1$.


La observaci\'on fundamental es que el conjunto de clases (o 
conjunto cociente) tiene una estructura natural de anillo:
\begin{enumerate}
\item Suma: $[n_1]+[n_2]=[n_1+n_2]$
\item Producto: $[n_1].[n_2]=[n_1. n_2]$
\end{enumerate}

Esta definici\'on de las operaciones es natural: para operar dos 
clases se eligen elementos {\bf cualesquiera} en cada una de ellas, 
se operan y la clase resultado es la clase del resultado.  Es 
esencial comprobar que la clase del resultado no depende de la 
elecci\'on de representantes de cada una de las clases, porque en 
caso contrario la operaci\'on no estar\'{\i}a bien definida (habr\'{\i} 
a varios resultados para una misma suma o producto). 

Se obtiene as\'{\i}   un 
anillo conmutativo, al que denotamos mediante $\Bbb Z_{m}$, y llamamos {\itshape
el anillo de clases de restos m\'odulo $m$.} 

?`En qu\'e condiciones es el anillo $\mathbb{Z}_{n}$ un cuerpo?  Gracias al
\hyperref[bezout]{teorema de Bezout} y dado que el $MCD(k,p)=1$ si
$p$ es primo y $k<p$, sabemos
que existen enteros $a$ y $b$ tales que $a\cdot k+b\cdot p=1$ de forma que
tomando restos m\'odulo $p$ se obtiene que el inverso de $[k]$ es $[a].$


El c\'alculo de potencias en los anillos de clases de restos $\mathbb{Z}_m$ es, 
computacionalmente, muy eficiente ya que podemos efectuar todas las operaciones
m\'odulo $m$. Al discutir  un m\'etodo para
calcular \hyperref[potencias]{potencias} de elementos de un anillo ya vimos que
para los anillos de clases de restos el algoritmo utilizado era  especialmente
eficiente.


Estos c\'alculos, potencias de un elemento en un anillo de clases de restos,
son, importantes en \hyperref[rsa]{criptograf\'{\i}a}, y concretamente
en el m\'etodo llamado RSA que es uno de los m\'as utilizados en la pr\'actica.
En el fundamento teórico de la criptograf\'{\i}a RSA est\'a el llamado
{\itshape teorema de Fermat-Euler}, un resultado bastante elemental pero
enormemente importante en aritm\'etica y sus aplicaciones.

\subsection{Teorema de Fermat-Euler}\label{peq}

Sea $[a]\in \mathbb{Z}_p$ una clase de restos no nula m\'odulo un entero primo,
y definimos la funci\'on $M_{[a]}:\mathbb{Z}_p\to \mathbb{Z}_p$ definida por
multiplicaci\'on por $[a]$: $M_{[a]}([b])=[a]\cdot[b]$. Iterando la funci\'on
$M_{[a]}$, dado que el conjunto de
clases de restos es finito, llegaremos, partiendo de una clase $[b]$ no nula 
cualquiera, necesariamente a un elemento ya visitado, es decir,
$M_{[a]^t}([b])=M_{[a]^s}([b])\  ([a]^t\cdot [b]=[a]^s\cdot [b])$ con $s<t$.
Como $[a]$ es invertible, la igualdad entre par\'entesis implica $[a]^{t-s}\cdot
[b]=[b]$, y como $[b]$ tambi\'en es invertible obtenemos finalmente
$[a]^{t-s}=[1].$

En resumen, para todo elemento no nulo $[a]\in \mathbb{Z}_p$ existe un entero
positivo $n_{[a]}$ m\'{\i}nimo tal que $[a]^{n_{[a]}}=[1].$ Es claro, entonces, 
que la iteraci\'on de la funci\'on $M_{[a]}$ termina siendo la funci\'on
identidad.

\begin{ejer}
\begin{enumerate}
 \item En una hoja de {\sage} define \lstinline|p=nth_prime(2013)|, y escribe un
programa para calcular el {\itshape conjunto} formado por los enteros 
$n_{[a]}$ con $[a]$ recorriendo los elementos no nulos de $\mathbb{Z}_p$. 

\item El m\'{\i}nimo com\'un m\'ultiplo de todos los enteros $n_{[a]}$ es un
entero $N$ tal que $[a]^N=[1]$ para toda clase $[a]\ne [0].$  Encuentra, en
$\mathbb{Z}_p$ como en el apartado anterior, el m\'{\i}nimo $N$ con esa
propiedad. 


\item Consideramos ahora un m\'odulo compuesto, por ejemplo
\lstinline|m=nth_prime(2013)*nth_prime(2014)|, y queremos estudiar el mismo tipo
de resultados.

Observar que las clases invertibles no son ahora todas las no nulas, 
sino las clases de enteros
$[k]$ tales que $k$ es primo con $m$. ?`Por qu\'e? 	

?`C\'omo se modifican los resultados, para m\'odulo primo,  anteriores para el
caso
de m\'odulo compuesto?

\item Escribe un programa que calcule las clases en $\mathbb{Z}_m$  que son
su propio inverso para la multiplicaci\'on, es decir, clases $[a]$ tales que
$[a]\cdot [a]=[1]$.  Utiliza el programa para tratar de caracterizar esas clases
en t\'erminos de la factorizaci\'on de $m$ como producto de primos.



 
 \end{enumerate}
\end{ejer}

\section{Fibonacci}\label{fibon}

La sucesi\'on de Fibonacci es la de n\'umeros enteros  definida por
\[F_0=0,\ F_1=1,\ F_m=F_{m-1}+F_{m-2},\]
\noindent y forma parte de una gran familia de sucesiones que decimos que
est\'an {\itshape definidas recursivamente.} En {\sage} disponemos de la
funci\'on \lstinline|fibonacci(m)| que, como es esperable,  devuelve el n\'umero
de Fibonacci $m$-\'esimo.  En los primeros ejercicios de esta secci\'on
implementamos diversos m\'etodos para calcular $F_m$ y la funci\'on
\lstinline|fibonacci(m)| de {\sage} nos puede servir para comprobar los
resultados obtenidos. 

\pagebreak[3]

\begin{ejer}
 \begin{enumerate}
  \item Programar una funci\'on recursiva que dado un entero $m$ devuelva el
$m$-\'esimo n\'umero de Fibonacci $F_m$.  Este ejercicio se ha resuelto como un
ejemplo en la p\'agina \pageref{fibor}.
  \item Programar una funci\'on iterativa  que dado un entero $m$ devuelva
el $m$-\'esimo n\'umero de Fibonacci $F_m$. Este ejercicio tambi\'en se ha
resuelto como un
ejemplo, ahora  en la p\'agina \pageref{fiboi}.
  \item \label{fib-matriz}Observa que es f\'acil obtener el vector
$(F_{k+1},F_k)^t$
multiplicando el par $(F_{k},F_{k-1})^t$ por una cierta matriz $\mathbf{A}$ que
no depende de $k$. Entonces, obtendr\'{\i}amos el vector $(F_{k+1},F_k)^t$
multiplicando $(1,0)^t$ por una potencia adecuada de la matriz $\mathbf{A}$.

Implementa esta manera de calcular la sucesi\'on de Fibonacci usando el
m\'etodo r\'apido para calcular potencias (ver p.~\pageref{potencias}). 
 
 \item Hay f\'ormulas que permiten calcular el $m$-\'esimo número de Fibonacci
$F_m$ sin haber calculado los anteriores. Una tal f\'ormula, debida a Binet
(1843),  
afirma que
\[F_m:=\frac{\phi^m-(1-\phi)^m}{\sqrt{5}},\]
con $\phi$ el {\itshape n\'umero áureo} $\frac{1+\sqrt{5}}{2}$. Observa que,
aunque la parte derecha de la f\'ormula es una expresi\'on complicada que
involucra ra\'{\i}ces cuadradas de $5$, la f\'ormula est\'a afirmando que el
resultado es un n\'umero entero para todo valor de $m$, es decir, que las
ra\'{\i}ces se van a cancelar al operar.

    En {\sage}  podemos operar de manera exacta, es decir, de forma {\itshape
simb\'olica} como opuesta a {\itshape numérica},  con expresiones como~$\phi$.
Basta tomar  \lstinline|sqrt(5)|, la ra\'{\i}z cuadrada de $5$, y usar las
operaciones aritm\'eticas habituales. Se puede forzar a {\sage} a desarrollar,
{\itshape quitar los par\'entesis},   una expresión como $F_m$ usando la función
\lstinline|expand| (ver p\'ag. \pageref{expand}). 

\begin{enumerate}
\renewcommand{\theenumi}{}%%Para quitarlas en las refs-cruzadas
\item\label{met-fibo-a} Comprueba que $F_m$ es el n\'umero de Fibonacci
$m$-\'esimo para $20$ valores de $m$ mayores que $10000$. Estudia la eficiencia
de esta f\'ormula como medio para calcular el n\'umero de Fibonacci $m$-\'esimo.

\item Define una función \lstinline|fibonacci_num(m,d)| que devuelva el n\'umero
$F_m$
calculado, aproximadamente,  como n\'umero real con $d$ d\'{\i}gitos de
precisi\'on.

\item Mediante las funciones \lstinline|floor(x), ceil(x)| o
\lstinline|round(x)|, usa la aproximaci\'on de $F_m$ dada por la f\'ormula de
Binet 
para calcular el valor exacto entero de $F_m$.

\item Estudia la eficiencia de este m\'etodo, comparando con el m\'etodo del
apartado
\ref{met-fibo-a} (siempre para valores de $m$ mayores que $10000$), y los
errores que pueden
aparecer al usar n\'umeros reales con $d$ d\'{\i}gitos de precisión. 
\end{enumerate}

\item Dado un n\'umero de Fibonacci bastante grande $F_m$ queremos calcular el
lugar que ocupa en la sucesi\'on, es decir, dado un $F_m$ queremos obtener el
$m$.
\begin{enumerate}
\item Define, usando la construcci\'on iterativa de la sucesi\'on de Fibonacci, 
una funci\'on que reciba como argumento  $F_m$ y devuelva $m$.
    
\item \label{fib-log} Cuando $m$ crece, el segundo sumando (de hecho es un
{\itshape restando})
en el numerador del $F_m$ del ejercicio anterior tiende a cero. Explica el
motivo. 

Entonces, $F_m$ es aproximadamente igual a $\phi^m/ \sqrt{5}$,  y esto
podr\'{\i}a servir para calcular $m$ dado un $F_m$ suficientemente grande.
Implementa un tal m\'etodo, y estudia su grado de validez comparando los valores
de $m$ que produce para variados $F_m$ calculados a partir de emes dados.
\end{enumerate}
\end{enumerate}
\end{ejer}

\subsection{Algunas propiedades de la sucesi\'on de Fibonacci}


\subsubsection{F\'ormulas}



Los n\'umeros de Fibonacci tienen una enorme cantidad de relaciones entre ellos,
y muchas de ellas se pueden demostrar por inducci\'on. Algunas de las m\'as
simples son las siguientes:

\begin{enumerate}
 \item\hfill $F_0+F_1+F_2+\dots+F_m=F_{m+2}-1,$\hfill\mbox{}

 \item\hfill $F_1+F_3+F_5+\dots+F_{2m-1}=F_{2m},$\hfill\mbox{}
 
 \item\hfill $F_0-F_1+F_2-F_3\dots-F_{2m-1}+F_{2m}=F_{2m-1}-1,$\hfill\mbox{}
 
 \item\hfill $F_0^2+F_1^2+F_2^2+\dots+F_m^2=F_{m}F_{m+1},$\hfill\mbox{}
 
 \item\hfill $F_{m-1}F_{m+1}-F_m^2=(-1)^m.$\hfill\mbox{}
 \end{enumerate}

 \begin{ejer}
Es f\'acil, y debes hacerlo,  escribir programas que comprueben, hasta un $m$
muy grande, que cualquiera de estas f\'ormulas es correcta, pero lo que
ser\'{\i}a m\'as interesante es encontrar m\'etodos para generar
autom\'aticamente y demostrar  f\'ormulas de esta clase.
\end{ejer}

\subsubsection{Divisibilidad}
\begin{ejer}
\begin{enumerate}
\item?`Qu\'e n\'umeros de Fibonacci son pares??`Cu\'ales son divisibles por $3$?
?`Y por $5$? ?`Qu\'e n\'umeros de Fibonacci son divisibles por $7$?

\item ?`Qu\'e podr\'{i}amos decir acerca de los divisores de un n\'umero de
Fibonacci que son tambi\'en n\'umeros de Fibonacci? 

\item ?`Hay n\'umeros de Fibonacci que sean primos?

\item?`Es (experimentalmente) cierto que si un n\'umero de Fibonacci $F_m$  es
primo entonces $m$ debe ser primo?

\item ?`Qu\'e relaci\'on hay entre el MCD de dos n\'umeros de Fibonacci,
$F_{m_1}$ y
$F_{m_2}$,  y el MCD de $m_1$ y $m_2$?

\item La sucesi\'on de Fibonacci se puede definir, mediante la misma
recursi\'on,  para las clases de restos m\'odulo un entero $m$, es decir, en el
anillo $\mathbb{Z}_m$. Como hay un n\'umero finito de clases de restos la
sucesi\'on deber\'{\i}a repetirse a partir de un valor, es decir, deber\'{\i}a
ser peri\'odica. 

Estudia el per\'{\i}odo en funci\'on de $m$, es decir, el m\'{\i}nimo $k>0$ tal
que 
\begin{center}
\lstinline|F${}_k$%m==0,F${}_{k+1}$%m==1|,
\end{center}
\noindent y trata de encontrar alg\'un orden en ese caos.
\end{enumerate}
\end{ejer}
\subsubsection{Curiosidad}
\begin{ejer}
La suma de la serie 
\[\sum_{k=2}^{\infty}\frac{F_k}{10^k}\]
\noindent es un n\'umero racional con un denominador menor que $100$. Calcula el
valor m\'as razonable para esa fracci\'on. 
\end{ejer}

\section{Algoritmo de Euclides}\label{eucl}

El algoritmo de Euclides para calcular el {\itshape m\'aximo com\'un divisor} de
dos enteros es ciertamente el m\'as antiguo que conocemos.

Dados dos enteros positivos $a_0$ y $b_0$  su 
m\'aximo com\'un divisor (MCD) es el mayor de sus divisores (positivos)
comunes. 

Obsérvese que hay un número finito de divisores (positivos) comunes, pues todos
están 
comprendidos entre $1$ y $\min(a,b)$.

% Para que esta definici\'on tenga sentido debemos saber que hay un 
% n\'umero finito de divisores comunes, ya que si hubiera infinitos no 
% tendr\'{\i}a sentido hablar del mayor. Esto es cierto ya que los 
% divisores comunes est\'an comprendidos entre $1$  y $\min(a,b)$.

El algoritmo de Euclides es un m\'etodo para calcular el MCD. 
Supongamos dados dos enteros positivos, $a_0$ y $b_0$, y queremos 
calcular su MCD.

\begin{enumerate}
\item Un primer paso ser\'a dividir el mayor,  por ejemplo
$a_0$, entre el otro. Si el resto es $0$ entonces $a_0$ es 
m\'ultiplo de $b_0$  y el MCD es $b_0$.

\item Si no, tendremos
$ a_0 = b_0 \cdot c_0+r_0$  con $ b_0>r_0>0$.  Esta \'ultima condici\'on 
hace que se pueda dividir  $b_0$ entre $r_0$, para obtener $b_0=r_0 \cdot
c_1+r_1$.

\item Si $r_1=0$  podemos sustituir $b_0$ en la primera divisi\'on y obtenemos
$a_0=r_0 \cdot( c_1 \cdot c_0 +1)$ de forma que $ b_0$ y $a_0$ son m\'ultiplos
de $r_0$. Falta 
comprobar que $r_0$ es el MCD. 

Si no lo fuera, existir\'\i a un divisor com\'un $k$ mayor que $r_0$.
Tendremos entonces $k\cdot k ^\prime =c_1 \cdot r_0$ y 
$k\cdot k^{\prime \prime} =(c_0\cdot c_1+1) \cdot r_0$. Sustituyendo se obtiene
$k \cdot k^{\prime \prime} =c_0  \cdot k \cdot k ^\prime +r_0$ y finalmente
$k( k^{\prime \prime}-c_0 \cdot  k^\prime )=r_0$ de forma que $k$  no es el 
MCD.

\item Si $r_1 \ne 0$ hay que repetir el proceso, dividiendo $r_0$ entre 
$r_1$, etc.  

\item El algoritmo finaliza, despu\'es de un n\'umero finito de divisiones, 
porque los restos sucesivos son cada vez m\'as 
peque\~nos pero siempre positivos o nulos. Cuando se alcance un 
resto nulo, el anterior es el MCD.

\item {\sc Ejercicio}: Demuestra esta \'ultima afirmaci\'on  por inducci\'on.

\end{enumerate}


Una de las propiedades m\'as \'utiles del MCD es el teorema de Bézout: {\itshape
Si $d$ es el MCD de $a_0$ y $b_0$, entonces 
existen enteros $u$ y $v$ tales que $d=u\cdot a_0+v\cdot b_0$.}
\label{bezout}
\bigbreak
Se puede demostrar  el teorema de Bézout como consecuencia 
del algoritmo de Euclides: 

Supongamos, por ejemplo, que hemos hecho cuatro divisiones:
\begin{equation}\notag
 \begin{aligned}
  a_0 &= b_0\cdot c_0+r_0\\
  b_0 &= r_0\cdot c_1+r_1\\
  r_0 &= r_1\cdot c_2+r_2\\
  r_1 &= r_2\cdot c_3 +0,
  \end{aligned}
\end{equation}

\noindent de forma que $r_2$ es el MCD de $a_0$ y $b_0$ ($a_0\ge b_0$). Ahora
podemos
escribir, despejando los restos y sustituyendo, 
\[r_2=r_0-r_1\cdot c_2=(a_0-b_0\cdot c_0)-(b_0-(a_0-b_0\cdot c_0)\cdot c_1)\cdot
c_2=
a_0(1+c_1\cdot c_2)+b_0(-c_0-c_2-c_0\cdot c_1\cdot c_2),\]

\noindent de forma que, en este caso en que hacemos $4$ divisiones al aplicar el
algoritmo,  $u=1+c_1\cdot c_2$ y $v=-c_0-c_2-c_0\cdot c_1\cdot c_2$.



Debe ser claro que si necesitamos m\'as de cuatro divisiones para llegar al MCD,
el procedimiento para calcular los coeficientes $a_0$ y $b_0$ en el teorema de
Bézout va a funcionar como en el caso de cuatro.

\subsection{M\'{\i}nimo com\'un m\'ultiplo}

Dados dos enteros positivos, $a_0$ y $b_0$, su {\itshape m\'{\i}nimo com\'un
m\'ultiplo} (MCM) es el menor de sus m\'ultiplos comunes.  Los dos enteros
tienen una infinidad de m\'ultiplos comunes, por lo menos todos los enteros de
la forma $k\cdot a_0\cdot b_0$, pero todo conjunto no vac\'{\i}o de enteros
positivos tiene un menor elemento, y, por tanto, el MCM est\'a bien definido.


Una manera f\'acil de calcular el MCM de dos enteros es, debido a que podemos
calcular su MCD mediante el algoritmo de Euclides,  usando el hecho de que el
producto de dos enteros positivos cualesquiera es igual al producto de su MCD
por su MCM.




\subsection{MCD en {\sage}}

En {\sage} disponemos de dos instrucciones principales relacionadas con el MCD:
\lstinline|gcd($a$,$b$)| y \lstinline|xgcd($a$,$b$)|. La primera devuelve el MCD
de los
dos enteros y la segunda una $3$-upla $(d,u,v)$ en la que el primer elemento es
el MCD de $a$ y $b$,  y los otros dos son enteros tales que $d=u\cdot
a+v\cdot b$, es decir, es decir se calculan un par de enteros tales que se
verifica el teorema de Bézout. 

Los enteros $u$ y $v$ no son \'unicos, de hecho, la ecuación $ax+by=d$, con $d$
el MCD de $a$ y $b$, tiene soluciones
$$x=u+(b/d)k,\ y=v-(a/d)k,$$
para alg\'un $k$ entero, con $u$ y $v$ una solución particular, por ejemplo la
dada por el algoritmo de Euclides.

Estas dos funciones de {\sage} tambi\'en pueden aplicarse a polinomios en una
variable con coeficientes racionales, ya que tales polinomios admiten divisi\'on
con resto y, en consecuencia, el algoritmo de Euclides es v\'alido para ellos. 

\

\begin{ejer}
\begin{enumerate}
 \item Programa el algoritmo de Euclides en forma iterativa y en forma
recursiva. 
 \item Programa, en forma iterativa y en forma recursiva, el algoritmo,
explicado m\'as arriba,  que nos
da los coeficientes $u$ y $v$ en el teorema de Bézout.
\item El algoritmo iterativo para calcular los coeficientes que aparecen en el
teorema de Bézout se puede organizar mucho mejor usando multiplicaci\'on de
matrices. Esto es, hasta cierto punto, an\'alogo al c\'alculo de los t\'erminos
de la \hyperref[fib-matrices]{sucesi\'on de Fibonacci} usando tambi\'en
multiplicaci\'on de matrices. 

Programar el algoritmo de esta manera.


\item Llamamos {\itshape ecuaci\'on diof\'antica lineal con dos inc\'ognitas a
una de la forma 
\[a\cdot x+b\cdot y=c,\]
\noindent donde $a,b,c$ son enteros dados,  y $x$ e $y$, las soluciones, deben
ser tambi\'en enteros tales que se cumple la ecuaci\'on. }

El teorema de Bézout permite resolver esta clase de ecuaciones. Escribe c\'odigo
tal que dados los coeficientes $a,b$ y $c$ devuelva las infinitas soluciones de
la ecuaci\'on diof\'antica como una f\'ormula en funci\'on de un par\'ametro $k$
entero.

\item Escribe una funci\'on de {\sage} tal que dados enteros $a$ y $b$ devuelva
el n\'umero de divisiones que hay que hacer al aplicar el algoritmo de
Euclides. 

Queremos estudiar, experimentalmente,  la siguiente cuesti\'on: {\itshape
``Encontrar, en funci\'on
de $k$, el n\'umero m\'aximo de divisiones que hay que hacer para calcular el
MCD de un n\'umero $a$ de m\'as de $k$ cifras y un n\'umero $b$ de exactamente
$k$ cifras''.} 
\end{enumerate}
\end{ejer}


\section{N\'umeros primos}\label{primos}

Un n\'umero $p>1$ es primo si sus \'unicos divisores son $\pm 1$ y $\pm p$. Uno
de los teoremas fundamentales, ya conocido por Euclides, afirma que {\itshape
todo entero $\ge 2$ es el producto, \'unico salvo el orden, de un n\'umero
finito de enteros primos}.

Tambi\'en Euclides sab\'{\i}a que hay infinitos n\'umeros primos, con el
siguiente argumento:
\begin{quotation}
``{\itshape Si hubiera un n\'umero finito de primos $p_1,p_2,\dots,p_n$ el
entero $p_1\cdot p_2\cdot \dots \cdot p_n+1$ no ser\'{\i}a divisible por ninguno
de ellos luego ser\'{\i}a, gracias al teorema de factorizaci\'on, 
necesariamente primo.}''
\end{quotation}

Si observamos la sucesi\'on de los primos inmediatamente vemos que seg\'un
avanza los huecos entre primos se hacen, {\itshape en general},  cada vez m\'as
grandes. Dicho de otra manera, podemos generar listas tan grandes como queramos
de enteros consecutivos todos compuestos:
\[n!+2,n!+3,n!+4,\dots,n!+n. \]

Entonces, uno de los problemas centrales, en la teor\'{\i}a de los n\'umeros
primos, es el de {\itshape la distribuci\'on de los n\'umeros primos dentro
del conjunto de los enteros,} que podemos entender como la determinaci\'on del
n\'umero, aproximado,  de primos en el intervalo $[N_1,N_2]$. 


Si definimos $\pi(x)$ como el n\'umero de primos menores o iguales que el
n\'umero real $x$ el {\itshape teorema de los n\'umeros primos}, conjeturado por
Legendre y Gauss,  y demostrado por Hadamard y de la Vall\'ee-Poussin, propone
que 
\[\lim_{n\to \infty}\frac{\pi(x)}{x/\mathrm{ln}(x)}=1.\]

Este resultado, y otros muchos en teor\'{\i}a de n\'umeros, se demostr\'o usando
t\'ecnicas de c\'alculo diferencial para funciones de una variable compleja, y
todav\'{\i}a es la demostraci\'on m\'as sencilla aunque hay otras.

El teorema nos informa, con una aproximaci\'on bastante buena,  de que el
n\'umero de primos $\le x$ es $x/\mathrm{ln}(x)$. 

Una de sus consecuencias es que el primo $n$-\'esimo $p_n$ est\'a pr\'oximo al
real $n\cdot \mathrm{ln}(n)$, aunque es siempre mayor (Rosser, 1938).
\subsection{N\'umeros primos en {\sage}}

Algunas de las funciones o m\'etodos disponibles en {\sage} para tratar con
n\'umeros primos: 
\begin{enumerate}
 \item En primer lugar tenemos el m\'etodo \lstinline|.is_prime()|, que devuelve
\lstinline|True| o \lstinline|False| 
seg\'un el entero, de {\sage},  al que se aplique sea primo o no. 
\item El m\'etodo \lstinline|.next_prime()| devuelve el primo siguiente al
entero al que se aplica. Sirve, por ejemplo, para localizar sucesiones grandes
de enteros consecutivos que sean todos compuestos.
\item La funci\'on \lstinline|prime_range(N_1,N_2)| devuelve la lista de los
primos contenidos en el intervalo $[N_1,N_2)$.
 
 \item La funci\'on \lstinline|nth_prime(m)| devuelve el primo, $p_m$,  
que ocupa el lugar $m$-\'esimo en la sucesi\'on de los n\'umeros primos.
 
 \item La funci\'on \lstinline|primes(N_1,N_2)| es un iterador sobre la lista de
los primos. Sirve para definir un bucle que itere sobre los enteros primos en el
rango $[N_1,N_2)$  utilizando poca memoria RAM.
\item Ya discutimos, en la p\'agina \pageref{factoriz}, c\'omo trabajar con las
factorizaciones de un entero como producto de primos.
 
 
\end{enumerate}





\subsection{Algunos problemas}
Dentro de la teor\'{\i}a de los n\'umeros primos hay bastantes conjeturas,
resultados probablemente ciertos pero para los que no se conoce demostraci\'on,
y resultados, algunos de demostraci\'on extremadamente dif\'{\i}cil, pero en
muchos casos con un enunciado muy simple. 


\bigskip


A continuaci\'on hay una peque\~na muestra, y para los problemas
marcados con un asterisco {\sc se propone que se elaboren programas} para
estudiarlos. 




\begin{enumerate}
 
 
 \item {\sc Conjetura de Goldbach}:
 
 \begin{enumerate}
 \item (Goldbach fuerte $^*$) Todo n\'umero par, mayor que $2$, es la suma de
dos
n\'umeros primos. 
 
 \item (Goldbach d\'ebil $^*$) Todo n\'umero impar mayor que cinco es la suma de
tres
n\'umeros primos. Esta conjetura parece que ha sido finalmente probada. 


Puedes leer el comienzo de este
\href{http://150.244.21.37/PDFs/TNUME/Harald.pdf}{art\'{\i}culo}, en el que el autor de la
demostraci\'on, el matem\'atico peruano 
Harald Andr\'es Helfgott, expone las ideas b\'asicas. Lo que nos interesa
aqu\'{\i} es que la demostraci\'on tiene dos partes:
\begin{enumerate}
\item Para n\'umeros impares menores que $10^{29}$ se comprob\'o mediante
ordenador.
\item Para los mayores que $10^{29}$ Helfgott mejor\'o  argumentos conocidos
hasta llegar a demostrar la conjetura para todos ellos.
\end{enumerate}
\item Tambi\'en se ha probado (Chen, 1973) que todo n\'umero par {\itshape
suficientemente grande} es la suma de dos primos o bien de un primo y un
producto de dos primos (por definici\'on un {\itshape semiprimo}).

\item (Conjetura de Lemoine)  Todo n\'umero impar mayor que cinco es la suma de
un n\'umero primo y 
de un semiprimo par.


\item (Conjetura de Sun $^*$) Todo n\'umero impar mayor que tres es la suma de
un
n\'umero primo impar 
y un producto de dos enteros consecutivos.

\end{enumerate}

\item {\sc Primos gemelos}:
 
 \begin{enumerate}
 \item ($^*$) Existen infinitos pares de n\'umeros primos de la forma $(p,p+2)$.
 
 \item Al comienzo de 2013 se ha presentado (Zhang)  una demostraci\'on de que
hay infinitos pares de primos que distan menos que $70$ millones. Esto est\'a
bastante lejos de la conjetura, pero en este momento parece que la cota, $70$
millones, ya ha sido rebajada a un n\'umero del orden de $4000.$
 
 \item ($^*$) Se sabe que todos los pares de primos gemelos $(p,p+2),\ p\ge 5$
son de
la forma $(6n-1,6n+1)$ para alg\'un $n$. Esto se debe a que todo primo mayor o
igual a cinco es de la forma $6n-1$ o de la forma $6n+1.$
 
\item ($^*$) {\sc Ejercicio}: encontrar todos los enteros $p$ tales que
$p,p+4,p+6,p+10,p+12,p+16,p+22$ sean todos primos.
 
 
 
 
 \end{enumerate}
 
 \item{\sc Primos como valores de polinomios }:
 \begin{enumerate}
 \item ($^*$) Dirichlet demostr\'o que los polinomios  $an+b$, verificando
$MCD(a,b)=1$, toman infinitos valores primos, es decir, {\itshape
 hay infinitos primos en las  progresiones aritm\'eticas en las que no todos los
elementos son m\'ultiplos de un entero $d$.}
 
 Por ejemplo,  es un ejercicio f\'acil,  usando la misma idea que la
demostraci\'on de Euclides
de la existencia de infinitos primos, demostrar que hay infinitos primos en la
progresi\'on aritm\'etica $4n+3$, pero el caso general es bastante dif\'{\i}cil.
 
 \item ($^*$) No se sabe si hay infinitos primos de la forma $n^2+1$. Sin
embargo se
sabe (Iwaniec, 1978)
que hay infinitos enteros de la forma  $n^2+1$ que tienen a lo m\'as dos
factores primos.
 \item No se conoce ning\'un polinomio de una variable, con coeficientes enteros
 y de grado mayor que uno, que tome infinitos valores primos.
 \item El polinomio $n^2+n+41$ comienza bien, toma valores primos para $n$ entre
$0$ y $39$, pero no se sabe que tome infinitos valores primos.
 
 \item {\sc Conjetura de Bouniakovsky ($^*$)}: Sea $P(x)$ un polinomio con
coeficientes
enteros  irreducible (i.e. no se puede escribir como producto de dos polinomios
con coeficientes enteros) y  de grado al menos $1$. Sea  $d$ el mayor entero tal
que divide a todos los valores  $P(n),\ n\in \mathbb{Z}$. Si $d$ vale $1$
entonces $P(n)$ tiene infinitos valores primos.  
 
 
 \item ($^*$) Sin embargo, hay polinomios de dos variables como $x^2+y^2$ para
los que
se puede demostrar, usando el teorema de Dirichlet, que toman infinitos valores
primos. 

\item Hay polinomios con coeficientes enteros en $10$ variables tales que el
conjunto de  los valores positivos que toman, cuando las variables recorren los
enteros no negativos, es el conjunto de enteros primos.
\end{enumerate}
\end{enumerate}

\section{Enteros representables como sumas de cuadrados}\label{suma-cuadrados}

La suma de los cuadrados de dos enteros, $x^2+y^2$, es un entero no negativo y
queremos averiguar qu\'e enteros no negativos se pueden representar como una tal
suma de cuadrados. Es claro que, por ejemplo, $2=1^2+1^2$ pero $3$ no se puede
representar como una suma de cuadrados. En estos ejercicios se trata de generar
un n\'umero suficiente de ejemplos de enteros que se representan, o no, como
sumas
de cuadrados,  y de obtener conjeturas razonables acerca de la condici\'on
(condiciones) que debe verificar un entero $n$ para que se pueda representar
como una suma de cuadrados.

\begin{ejer}
\begin{enumerate}
\item ?`Qu\'e n\'umeros primos se pueden representar como suma de dos cuadrados?
Mediante un programa adecuado no es dif\'{\i}cil ver cu\'al debe ser la
respuesta, aunque no es tan f\'acil demostrarla.

\item ?`Qu\'e enteros positivos se pueden representar como una suma de
cuadrados?

Gracias a las identidades

\begin{equation}\notag
 \begin{aligned}
  (x_1^2+y_1^2)(x_2^2+y_2^2)&=(x_1x_2\pm y_1y_2)^2+(x_1y_2\mp y_1x_2)^2\\
  2(x_1^2+y_1^2) &=(x_1+y_1)^2+(x_1-y_1)^2
 \end{aligned}
\end{equation}

\noindent no es dif\'{\i}cil responder a esta pregunta usando la respuesta de la
anterior y el hecho de que todo entero es un producto, esencialmente \'unico, 
de potencias de primos. 

\item ?`Qu\'e enteros positivos se pueden representar como una suma de tres
cuadrados? ?`Y como una suma de cuatro cuadrados?
\end{enumerate}
\end{ejer}
\section{Desarrollos decimales}\label{decimales}
Dada una fracci\'on reducida $n/d$, con $n$ y $d$ enteros positivos primos entre
s\'{\i}, podemos asociarle un n\'umero real, decimal con infinitas cifras, 
mediante el algoritmo de divisi\'on con resto de n\'umeros enteros.
\begin{enumerate}
\item Para empezar, si $n>d$,  separamos la parte entera  dividiendo
$n=c\cdot d+r$,  de forma que $\frac{n}{d}=c+\frac{r}{d}$ con $c$ la parte
entera de la fracci\'on. Podemos continuar con el desarrollo decimal de $r/d$,
es decir, suponiendo que $n<d.$
\item Si el denominador $d$ es de la forma $2^a\cdot 5^b$, llamemos $n$ al
m\'aximo de $a$ y $b$,  obtenemos un n\'umero entero multiplicando $r/d$ por
$10^n$, y, por tanto el desarrollo decimal de $r/d$ es  cero a partir del lugar
$n$-\'esimo despu\'es del punto decimal.

\item Supongamos ahora que $n<d$ y  $MCD(d,10)=1.$ ?`C\'omo podemos calcular la
primera cifra decimal de $n/d$? Es claro que la primera cifra decimal es la
parte entera de $(10\cdot n)/d$. 

Por ejemplo, la primera cifra decimal de $3/7$ es $4$ como vemos al realizar la
división entera de $3\cdot10=30$ por $7$, esto es
$30=4\cdot 7+2$.

?`Y la segunda cifra decimal? Habr\'a que restar a $(10\cdot n)/d$ su parte
entera, multiplicar por $10$ otra vez y quedarnos con la parte entera del
resultado. En el ejemplo, y dividiendo la división entera ya realizada por $7$,
encontramos $\frac{30}7-4=\frac27$, y continuaríamos con el desarrollo decimal
para $2/7$.

\item Defino entonces dos  funciones
\begin{enumerate}
\item \lstinline|f(r):=floor(10*r)|, que aplicada a un n\'umero racional menor
que $1$ produce la primera cifra de su desarrollo decimal.

 \item \lstinline|F(r):=10*r-floor(10*r)|, que aplicada a
un n\'umero racional $r=n/d$ produce otro n\'umero racional $r_1:=n_1/d<1$, con
el mismo denominador que $r$, que es la fracci\'on que contiene la informaci\'on
sobre las siguientes cifras decimales de $n/d.$
En particular, aplicando $f$ al racional $F(r)$ obtenemos la segunda cifra
decimal de $r$. 
\end{enumerate}

\item Como todas las fracciones que obtenemos al iterar $F$ tienen el mismo
denominador $d$, s\'olo puede haber un n\'umero finito de numeradores, enteros
entre $1$ y $d-1$, lo que sugiere que tiene sentido considerar la cuesti\'on en
el anillo de clases de restos m\'odulo $d$. 

\item ?`Qui\'en es el numerador $n_1$ de la fracci\'on $r_1$? Si escribimos
$10*n=c\cdot d+r$ y dividimos entre $d$ ambos miembros vemos que $n_1$ es el
resto $r$ de esta divisi\'on, de forma que cuando consideramos la funci\'on $F$
como actuando en los numeradores de fracciones que tienen $d$ como denominador,
se trata simplemente de la funci\'on, de $\mathbb{Z}_d$ en s\'{\i} mismo,  
definida mediante $\overline{F}([n]):=[10*n],$ y sabemos, gracias al teorema de
Fermat-Euler
(ver la subsecci\'on \ref{peq}), que existe un exponente $\nu$ tal que
$\overline{F}^{\nu}=Identidad.$

Esto demuestra que {\itshape los desarrollos decimales de fracciones $n/d$ con
$n<d$ y $MCD(d,10)=1$ son peri\'odicos.}

\item Por \'ultimo, si el denominador $d$ de la fracci\'on $n/d$ que estamos
estudiando es de la forma $2^a\cdot 5^b\cdot D$ con $MCD(D,10)=1$, y
multiplicamos la fracci\'on por $10^{\max(a,b)}$ obtenemos una expresi\'on de la
forma $C+\frac{N}{D}$ con $C$ un entero. Ya sabemos que el desarrollo decimal de
$N/D$ es peri\'odico, de manera que dividiendo por 
$10^{\max(a,b)}$ obtenemos que el desarrollo decimal de la fracci\'on original
tiene un primer bloque de la forma $C/10^{\max(a,b)}$ y a continuaci\'on
otro bloque repetido infinitas veces,  al que llamamos la {\itshape parte
peri\'odica del desarrollo}. 
\end{enumerate}

?`C\'omo podemos obtener la fracci\'on que corresponde a un cierto desarrollo
decimal peri\'odico?  Supongamos, por ejemplo, un desarrollo de la forma 
\[x:=2{.}34545454545\dots=:2{.}3\overline{45}.\]

Podemos escribir $x=\frac{23}{10}+\frac{0{.}\overline{45}}{10}$. Tomando 
$s=0{.}\overline{45}$, y restándoselo a \(100s=45{.}\overline{45}\)
se obtiene $s=\frac{45}{99}=\frac5{11}$ y así
\[x=\frac{23}{10}+\frac{\frac{5}{11}}{10}=\dots=\frac{129}{55}
\]


\begin{ejer}
\begin{enumerate}
 \item  Programa una funci\'on, o varias, para calcular el desarrollo decimal de
una fracci\'on $n/d$ reducida cualquiera.
 \item Programa una funci\'on para calcular la fracci\'on que corresponde a un
desarrollo decimal dado.
 \item ?`Qu\'e modificaciones habr\'{\i}a que hacer para que estos programas
funcionaran para fracciones de enteros expresados en el
\hyperref[bases-num]{sistema de numeraci\'on
de base $b$}?
 
\end{enumerate}


\end{ejer}

