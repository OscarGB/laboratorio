\section{Bibliograf\'{\i}a}



\begin{enumerate}
 \item {\sc Documentaci\'on producida por sagemath:} Los desarrolladores de
{\sage} producen y mantienen actualizada una cantidad inmensa de documentaci\'on
sobre el sistema. Puede verse completa en 
\href{http://www.sagemath.org/help.html#SageStandardDoc}{este enlace.}

Hemos seleccionado y la que nos parece m\'as interesante, de forma que no es
necesario decargarla. Hay zonas de estos documentos que todav\'{\i}a no han sido
escritas, es decir, la documentaci\'on parece estar siempre en desarrollo.
 
 Por otra parte, parte de estos documentos se refieren a partes de las
matem\'aticas mucho m\'as avanzadas, y que, por supuesto, no nos conciernen en
este curso.
 
 \begin{enumerate}
 \item En primer lugar tenemos el 
 \href{http://150.244.21.37/PDFs/SAGE-DOCS/SageTutorial.pdf}{{\itshape tutorial}} de
{\sage}, del que los tres primeros cap\'{\i}tulos son lectura muy recomendable.
Su contenido se puede solapar en parte con el del comienzo de estas notas, pero
encontrar\'eis all\'{\i} multitud de ejemplos cortos que ayudan a comenzar con
{\sage}.

\item  El \href{http://150.244.21.37/PDFs/SAGE-DOCS/sage-power.pdf}{documento {\tt
sage-power.pdf}} contiene informaci\'on m\'as avanzada sobre el uso de {\sage}.

\item Este otro
\href{http://150.244.21.37/PDFs/SAGE-DOCS/thematic_tutorials.pdf}{documento} contiene
informaci\'on sobre el uso de {\sage} en campos concretos de las matem\'aticas.
En particular, pueden ser \'utiles la secciones  {\tt 6.1.7} (teor\'{\i}a de
n\'umeros y criptograf\'{\i}a),  {\tt 6.1.1} (teor\'{\i}a de grafos) y el
cap\'{\i}tulo 5 (t\'ecnicas de programaci\'on). 
 \item En este \href{http://150.244.21.37/PDFs/SAGE-DOCS/constructions.pdf}{documento} se
explica c\'omo funcionan muchos de los objetos abstractos, grupos, anillos,etc.,
 que  se pueden construir y manipular en {\sage}. 
 \end{enumerate}
 
 \item {\sc Otros:}
 \begin{enumerate}
  \item Un \href{http://150.244.21.37/PDFs/INTRO/sagebook-FR.pdf}{libro} interesante de
introducci\'on a {\sage}. ``Desgraciadamente'' est\'a en franc\'es, pero los
ejemplos de c\'odigo sirven. 
  \item  Otro \href{http://150.244.21.37/PDFs/INTRO/sage_for_newbies_v1.23.pdf}{texto} con
aplicaciones, sobre todo,  a las matem\'aticas de la etapa preuniversitaria. 

\item  Este \href{http://150.244.21.37PDFs/INTRO/sage_for_undergraduates_color.pdf}{texto} se centra en  las matem\'aticas de los primeros cursos de Universidad. 

  
 \end{enumerate}
\item {\sc Python:} 

{\sage} incluye varios sistemas preexistentes de c\'alculo simb\'olico y el
{\itshape pegamento} que los hace funcionar juntos es Python. Cuando queremos
sacarle partido a {\sage} necesitamos utilizar Python para definir nuestras
propias funciones, que frecuentemente incluyen funciones  de {\sage}
dentro. Python es entonces tambi\'en el {\itshape pegamento} que nos permite
obtener de {\sage} respuestas a problemas que no estaban ya preprogramados.

\begin{enumerate}
 \item En primer lugar tenemos aqu\'{\i} un 
\href{http://150.244.21.37/PDFs/PYTHONpython3handson.pdf}{{\itshape tutorial}} de Python.
Utiliza la versi\'on $3$ del lenguaje, mientras que {\sage} todav\'{\i}a usa la
$2.7$. 
\item \href{http://150.244.21.37/PDFs/PYTHONpythonTutorial.pdf}{El {\itshape tutorial}}
escrito por el autor, Guido van Rossum, del lenguaje. Tambi\'en para la
versi\'on $3$ y bastante avanzado. 

\item  \href{http://150.244.21.37/PDFs/PYTHONthinkpython.pdf}{Una muy buena
introducci\'on} al lenguaje. Hay  una
\href{http://150.244.21.37/PDFs/PYTHONthinkpython-3.pdf}{versi\'on} del texto  que
utiliza Python $3$. Allen B. Downey es tambi\'en el autor de este
\href{http://150.244.21.37/PDFs/PYTHONthinkcomplexity.pdf}{libro}, en el que se aplica
Python a diversos problemas, como los aut\'omatas celulares, relacionados con la
{\itshape teor\'{\i}a de la complejidad}. Este \'ultimo texto contiene muchas
ideas y ejercicios interesantes,  y lo hemos usado  para preparar la parte final
del  curso.
\end{enumerate}

 

 
 
 
 
 
 
 
 \item {\sc Bibliograf\'{\i}a por materias:}
\begin{enumerate}
 \item Un \href{http://150.244.21.37/PDFs/SAGE-DOCS/prep_tutorials.pdf}{estupendo resumen}
del uso de {\sage} en matem\'aticas, sobre todo en c\'alculo y representaciones
gr\'aficas.
 \item \href{http://150.244.21.37/PDFs/CAVAN/calculus1-with-sage.pdf}{Un curso} de
c\'alculo diferencial que utiliza  {\sage} masivamente. No contiene c\'alculo
integral. 
 
 \item En este \href{http://www.sagemath.org/calctut/}{sitio
 \itshape{web}} (parte de la documentaci\'on oficial de {\sage}) puede verse otra
 introducci\'on al estudio del c\'alculo diferencial usando {\sage}.
 
 
 
 
 \item \href{http://150.244.21.37/PDFs/CAVAN/fcla/fcla-3.20-print.pdf}{Un curso} {\itshape
tradicional} de \'Algebra lineal que contiene una
\href{http://150.244.21.37/PDFs/CAVAN/fcla/fcla-3.20-sage-6.0-primer.pdf}{extensi\'on} sobre el
uso de {\sage} para resolver problemas. 
\item Un \href{http://150.244.21.37/PDFs/CRIPT/kohel-book-2008.pdf}{texto} bastante completo de 
criptograf\'{\i}a que utiliza {\sage} para realizar los c\'alculos. 

\item Un \href{http://150.244.21.37/PDFs/PROBA/probability.pdf}{curso} muy popular, llamado
{\sc chance}, de introducci\'on a la teor\'{\i}a de de probabilidades. Los
primeros cap\'{\i}tulos pueden ser de alguna utilidad cuando lleguemos al
cap\'itulo \ref{prob}.


\end{enumerate}


 
  
 
 
 
\end{enumerate}








\section{Otros}
 \begin{enumerate}
  \item Ya se \hyperref[chuletas]{mencionaron los chuletarios} de {\sage} como
forma r\'apida de acceder a la sintaxis de las principales instrucciones
preprogramadas. 
\item Tambi\'en se indic\'o en el pr\'ologo de las notas que este curso se basa,
en gran medida, en sus primeras versiones que mont\'o Pablo Angulo. Puedes ver 
el curso original en 
  \href{http://verso.mat.uam.es/~pablo.angulo/doc/laboratorio/index.html}{este
enlace}. 


  
  
  \item El \href{http://rosettacode.org/wiki/Rosetta_Code}{sitio {\itshape web}}
{\tt rosettacode.org} es especialmente \'util cuando se sabe ya programar en
alg\'un lenguaje pero se pretende aprender uno nuevo. Consiste en una serie
grande de problemas resueltos en casi todos los lenguajes disponibles,  en
particular, casi todos est\'an resueltos en los m\'as populares como $C$ y
Python.

A fin de cuentas, todos los lenguajes tienen bucles \lstinline|for| y
\lstinline|while|, bloques \lstinline|if| y recursi\'on. Hay entre ellos
peque\~nas diferencias en la sintaxis y en la forma en que se manejan las
estructuras de datos. 
  
  \item En el \href{http://projecteuler.net/}{sitio {\itshape web}} {\tt
projecteuler} pueden encontrarse los enunciados de casi $500$ problemas de
programaci\'on. D\'andose de alta en el sitio es posible ver las soluciones
propuestas para cada uno de ellos.  

Tal como est\'an enunciados muchos de ellos son dif\'{\i}ciles, ya que no se
trata \'unicamente de escribir c\'odigo que, en principio, calcule lo que se
pide, sino que debe ejecutar el c\'odigo en un ordenador normal de sobremesa, 
en un tiempo razonable, para valores de los par\'ametros muy altos. Es decir, lo
que piden esos ejercicios es que se resuelva el problema mediante un c\'odigo
correcto y muy eficiente. 

  
  
  
  \end{enumerate}

 
 
 
 
 

 
 
 