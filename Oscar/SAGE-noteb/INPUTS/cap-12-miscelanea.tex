
En este \'ultimo cap\'{\i}tulo, especie de caj\'on de sastre (?`caj\'on 
desastre?),  se incluyen ejercicios sobre temas sueltos que no encajan el los 
previos pero que creemos interesantes. No suele ser posible cubrirlos todos 
durante el curso y, de hecho,  esperamos ir a\~nadiendo otros nuevos en cursos 
sucesivos y cada curso seleccionar algunos.




\section{Aut\'omatas celulares}

%%Unodimensionales y bidimensionales RHG -- aleatoriedad cripto ---
%%Juego vida--avalanchas 
%%Hormigas
%%Peces 
%%RHG

%%CREADO 22-05-2013/11:10
Los aut\'omatas celulares consisten en un tablero, una configuraci\'on inicial 
del tablero y unas reglas que determinan como se pasa de una configuraci\'on a 
la siguiente, es decir, determinan la {\itshape evoluci\'on temporal} del 
aut\'omata. El tablero puede ser $1$-dimensional, $2$-dimensional, etc., y cada 
casilla, que debe estar coloreada para distinguirla de las vecinas, puede ser 
de 
un color elegido entre ile$k$ posibles, aunque frecuentemente $k=2$. Las reglas de 
evoluci\'on del aut\'omata dependen, para cada casilla,  de los colores que 
presentan las casillas vecinas. 

El aspecto quiz\'a m\'as interesante en los aut\'omatas celulares es que, 
aunque 
su evoluci\'on est\'a totalmente determinada por el tablero inicial y las 
reglas 
de evoluci\'on, y por tanto es un sistema completamente determinista, puede 
generar comportamientos complejos que muestran cierta (pseudo-)aleatoriedad. 

Otro aspecto interesante es que algunos aut\'omatas celulares muestran un 
{\itshape comportamiento reversible}, en el sentido de que la situaci\'on en el 
momento actual no s\'olo determina el comportamiento futuro sino tambi\'en el 
pasado. Como las leyes f\'{\i}sicas fundamentales son reversibles, el estudio 
de 
estos modelos merece la pena.

Puedes leer sobre la teor\'{\i}a de los aut\'omatas celulares en 
\href{http://150.244.21.37/PDFs/MISCE/ca-review.pdf}{este art\'{\i}culo}. 

\subsection{El Juego de la Vida}

Se trata de un aut\'omata celular bidimensional, que simula, hasta cierto 
punto, 
 la evoluci\'on de una {\itshape colonia de bacterias.} Puedes verlo en 
funcionamiento pinchando en el enlace en la esquina superior izquierda de esta  
\href{http://www.ibiblio.org/lifepatterns/}{p\'agina web.}

Tambi\'en puedes leer sobre el juego y algunas de las configuraciones iniciales 
m\'as interesantes en 
\href{http://150.244.21.37/PDFs/MISCE/Conway_life.pdf}{este art\'{\i}culo de la Wikipedia}. 


\begin{enumerate}
 \item Se trata de un juego con cero jugadores. El juego comienza en una
configuraci\'on inicial y evoluciona solo.
 \item Se juega en un tablero, como de ajedrez,  infinito con casillas vivas,
que ser\'an las blancas,  y casillas muertas las negras. La configuraci\'on
inicial, que suele tener un n\'umero finito de casillas vivas,  es arbitraria
(no la del tablero de ajedrez), y en cada etapa del juego las casillas pueden
permanecer en su estado o cambiar al otro de acuerdo con reglas precisas. 
 \item Las reglas que determinan el estado de una casilla en la etapa siguiente
dependen del estado en esa etapa de las ocho casillas vecinas: dos en
horizontal, dos en vertical y cuatro en las diagonales. Concretamente son las
siguientes:
 \begin{enumerate}
  \item Una casilla viva con menos (estricto) de dos vecinos vivos muere, de 
{\itshape aburrimiento}, en la
siguiente fase .
  \item Una casilla viva con dos o tres vecinos vivos se mantiene viva en la
siguiente fase.
  \item Una casilla viva con m\'as de tres vecinos vivos muere debido al
{\itshape stress que le causa su intensa vida social}.
  \item Una casilla muerta con exactamente tres vecinos vivos cambia a viva en
la siguiente fase porque {\itshape los tres vecinos vivos (!entre los tres, dos 
s\'olo
no pueden!) han procreado}.
 \end{enumerate}

 {\sc N\'otese}, y es importante para entender el proceso, que la
actualizaci\'on entre una fase y la siguiente se hace {\itshape de golpe} en 
todo el
tablero, es decir, si cuando pasamos de la etapa $n$ a la $n+1$ hemos cambiado
una casilla de muerta a viva ese cambio no influye en el resultado final en la
etapa $n+1$ que s\'olo depende del estado final en la etapa $n$. 
 
 En particular, la observaci\'on anterior obliga a mantener en memoria el estado
del tablero en la fase $n$ mientras vamos actualizando un tablero inicielmente
completamente muerto hasta obtener el estado $n+1$.  Podemos representar el
estado del tablero en cada fase mediante una matriz de ceros y unos.
 
 
\item En la pr\'actica hay que hacer que el tablero sea finito para poderlo
manejar en el ordenador, y  vamos a usar {\itshape condiciones de 
per\'iodicidad en el
borde}: cada casilla en el borde, por ejemplo en el lateral de la izquierda,
tiene cinco vecinos hacia el interior del tablero y los otros tres son los que,
pegados al lateral de la derecha, ocupan posiciones sim\'etricas de la casilla
en cuesti\'on y sus dos vecinos en vertical. 

Decimos que el juego se est\'a desarrollando en un {\itshape tablero toroidal}.
 
\item La idea de este juego es, claramente, que las reglas permitan estudiar la
evoluci\'on en el tiempo de una {\itshape colonia de casillas}, simulando 
organismos
vivos,  que {\itshape nacen, se reproducen  y mueren},  pero tambi\'en se 
autoorganizan en
{\itshape organismos pluricelulares} que pueden incluso {\itshape desplazarse} 
por el tablero.
 

\item Sorprende la enorme riqueza de comportamientos que se ha obtenido, pero
todav\'{\i}a sorprende m\'as que el juego es una {\itshape m\'aquina universal 
de
Turing} , lo que significa, m\'as o menos, que cualquier problema para el que
existe una soluci\'on algor\'{\i}tmica (i.e. problema que se puede resolver en
un ordenador) se puede resolver dentro del juego de la vida. Por ejemplo, es
posible codificar dos enteros binarios dentro de una configuraci\'on inicial de
forma que en un n\'umero finito de fases se alcance un estado, que ya no
var\'{\i}a,  y en el que se puede leer la suma o el producto de los enteros.
\end{enumerate}


\begin{ejer}
 
{\sc Programar} el juego de la vida usando estas instrucciones:

\begin{enumerate}
\item Conviene crear una funci\'on {\tt vecinos((n,m))} que devuelva los ocho
vecinos de cada casilla $(n,m)$ teniendo en cuenta que el tablero es un toro.
\item Una funci\'on {\tt siguiente(M)} que tome como argumento una matriz
cuadrada {\tt M}, $N\times N$ de ceros y unos,  correspondiente a la fase $N$
del tablero y devuelva la matriz correspondiente a la fase $n+1$. 
\item Una funci\'on {\tt itera\_hasta(siguiente,M,T)} que reciba como argumentos
la funci\'on {\tt siguiente},  la matriz del estado inicial del tablero $M$ y el
n\'umero de iteraciones $T$,  y devuelva una lista con entradas los primeros $T$
estados del tablero.
\item Representaci\'on gr\'afica:
\begin{verbatim}
 v = map(matrix_plot,itera_hasta(siguiente,M,20))
 animate(v).show(delay=50,iterations=10)
\end{verbatim}

\end{enumerate}

\end{ejer}

En esta hoja de {\sage} puedes ver una soluci\'on
\href{http://localhost:8080/home/admin/??}{\tt 121-MISCE-vida.sws}.

\subsection{Aut\'omatas celulares 1-dimensionales}

En este ejercicio vamos a estudiar una familia de aut\'omatas celulares de 
dimensi\'on uno. El {\itshape juego de la vida} es un aut\'omata celular de 
dimensi\'on dos y, por tanto, algo m\'as complicado que los que veremos en este 
ejercicio. El dato principal es una lista de ceros y unos que representar\'a el 
estado inicial del aut\'amata. En el juego de la vida, en lugar de una lista 
como estado inicial usamos una matriz.

El estudio sistem\'atico, mediante ordenador, de los aut\'omatas celulares fue 
comenzado por Stephen Wolfram\footnote{Seg\'un informaci\'on obtenida en Internet, quiz\'a no totalmente fiable, el personaje {\itshape Sheldon Cooper} en la serie {\itshape The big bang theory}  est\'a basado directamente en la vida real de Wolfram.}, creador del programa de c\'alculo simb\'olico, 
rival de {\sage}, Mathematica. Wolfram ha escrito un tocho de unas $1200$ 
p\'aginas, {\itshape A new kind of science}, que aunque un poco 
megaloman\'{\i}aco a veces, es,  sin embargo,  bastante interesante. La familia 
de aut\'omatas considerada en este ejercicio fu\'e definida y estudiada por 
\'el en los $80$. 

\begin{ejer}
\begin{enumerate} 
\item Define  una funci\'on de {\sage} {\tt vecinos(k,L)} que para cada entero 
$k$ entre $0$ y $len(L)-1$ devuelva la $3$-upla  formada por el elemento de 
$L$ anterior a {\tt L[k]}, el elemento {\tt L[k]} y el siguiente a {\tt L[k]}, 
teniendo en cuenta que el siguiente al \'ultimo debe ser el primero y el 
anterior al primero debe ser el \'ultimo. 
\item Consideramos  la lista de las $8$  $3$-uplas de ceros y unos que podemos 
formar:
\[T=[(0, 0, 0),
 (1, 0, 0),
 (0, 1, 0),
 (1, 1, 0),
 (0, 0, 1),
 (1, 0, 1),
 (0, 1, 1),
 (1, 1, 1)].
 \]

\item Cada entero  $0\le k \le 255$ se puede escribir en binario con $8$ bits 
(rellenando con ceros cuando haga falta). Dado un tal entero $k$ le podemos 
hacer corresponder, de esta manera,  una lista $B(k)$ de $8$ ceros o unos.
\item Define   una función de {\sage}  que reciba como argumento un entero 
$0\le 
k\le 255$ y devuelva un diccionario $D(k)$ con claves las $3$-uplas de las 
lista 
$T$ y, para cada $3$-upla tome valor el cero o uno que ocupa el mismo lugar en 
la lista $B(k)$ del apartado anterior. Observa que el orden en que hemos 
definido la lista $T$ de $3$-uplas influye en el diccionario obtenido. 
\item Define  una función de {\sage}  {\tt siguiente(L,k)} que reciba como 
argumentos una lista $L$ de ceros o unos y un entero $0\le k\le 255$ y devuelva 
otra lista de la misma longitud que $L$  que en el lugar $i$ tenga el valor que 
el diccionario $D(k)$ asigna a la $3$-upla de vecinos de {\tt L[i]}. Esta 
función es la que determina la evoluci\'on en el tiempo del aut\'omata.
\item Define  una función {\tt evolucion(L,k,N)} que devuelva la lista de 
listas, de longitud $N+1$,  que se obtiene al iterar, $N$ veces,  la función 
{\tt siguiente} partiendo del valor dado de $L$, es decir, debe devolver la 
lista
\[
[L,siguiente(L),siguiente(siguiente(L)),\dots,siguiente^N(L)].
\]

\item Podemos transformar la lista devuelta por {\tt evolucion(L,k,N)} en una 
matriz con $N+1$ filas y $len(L)$ columnas, y representar gr\'aficamente la 
matriz usando {\tt matrix\_plot}. Obtener esas representaciones gr\'aficas para 
los valores $k=18,30,50,110$, $N=256$ y la lista $L$ inicial formada por $128$ 
ceros, un uno y otros $128$ ceros. Observa las diferencias en las 
características de los gr\'aficos obtenidos.

 
 \end{enumerate}
\end{ejer}





\section{Percolaci\'on}

%%Cambios fase -- valores criticos RHG


Se llama {\itshape teor\'{\i}a de la percolaci\'on} a una serie de modelos 
matem\'aticos que tratan de captar los aspectos importantes de la situaci\'on 
real de un l\'{\i}quido que se filtra a trav\'es de un material poroso. La 
teor\'{\i}a forma parte de la teor\'{\i}a de la Probabilidad, ya que un 
ingrediente fundamental es una probabilidad $p$ fijada, y la misma en todos los 
puntos,  de que  el l\'{\i}quido se filtre de un punto a un punto vecino. 


Es interesante que la percolaci\'on, como modelo matem\'atico, presenta 
{\itshape fen\'omenos cr\'{\i}ticos}. Se puede demostrar matem\'aticamente, 
pero tambi\'en estudiar experimentalmente en el ordenador, que existe un valor 
cr\'{\i}tico $p_c$ de la probabilidad que define el modelo tal que si $p\le 
p_c$ no se produce con casi total seguridad percolaci\'on hasta infinito, 
mientras que si $p>p_c$ se produce con casi total seguridad percolaci\'on hasta 
infinito. Esto hace que el estudio de la percolaci\'on sea, en cierta medida, 
semejante al estudio de los cambios de fase de la materia en f\'{\i}sica 
(congelaci\'on, evaporaci\'on, etc.), para los que tambi\'en existen 
temperaturas cr\'{\i}ticas que separan los diferentes estados. 

Puedes leer m\'as sobre la percolaci\'on  en
\href{http://150.244.21.37/PDFs/MISCE/percolation.pdf}{este art\'{\i}culo}.
\par
\bigskip
\par
{\sc Descripci\'on del problema:}


\begin{enumerate}
\item Consideramos el ret\'{\i}culo $L$ formado por los puntos de coordenadas 
enteras, a los que llamamos nodos de $L$, en el plano $\mathbb{R} \times 
\mathbb{R}$ junto con los segmentos que unen cada uno de esos puntos a los 
adyacentes. Cada nodo $(n, m)$ tiene
cuatro adyacentes, o vecinos, $(n-1, m),\  (n, m-1),\  (n + 1, m)$  y $(n, m 
+ 1).$
\item Fijamos un n\'umero real $p\in (0, 1)$ y para cada uno de los cuatro 
vecinos del origen $(0, 0)$ lanzamos una moneda trucada con probabilidad $p$ de 
obtener cara y, en caso de obtener cara, decimos que un l\'{\i}quido que se 
genera en el
origen ha {\itshape percolado} al nodo en cuesti\'on.
\item Este proceso se repite para cada nodo al que el l\'{\i}quido ha 
percolado, 
y se trata de analizar los valores de
$p$ para los que el l\'{\i}quido {\itshape percola hasta infinito}.
\item Se puede considerar que \'este es un modelo de un l\'{\i}quido que se 
filtra a trav\'es de un s\'olido poroso, con {\itshape porosidad}
controlada por la probabilidad $p$, y tratamos de averiguar si el l\'{\i}quido 
{\itshape transpasa} o no. Por supuesto es un modelo matem\'atico muy 
idealizado 
y, por ejemplo, en {\itshape la realidad} el s\'olido no ser\'a infinito.
\item  Tambi\'en puede verse como un modelo de la transmisi\'on de una 
enfermedad contagiosa, que comienza en $(0, 0)$ y cuya probabilidad de contagio 
de una persona enferma a sus {\itshape pr\'oximas} es $p.$
\item  ?`Se podr\'{\i}a estudiar mediante un modelo similar la propagaci\'on de 
los incendios forestales? La dificultad radicar\'{\i}a en que la probabilidad 
de 
propagaci\'on de un punto a los pr\'oximos deber\'a depender de las condiciones 
locales (existencia de material combustible, viento en cada momento, 
orograf\'{\i}a, etc.) cerca del punto, y no parece f\'acil determinarla.

\end{enumerate}

\begin{ejer}
Programamos la percolaci\'on en un ret\'{\i}culo de cuadrados seg\'un las 
siguientes indicaciones:
\begin{enumerate}
\item Debemos mantener dos listas que ir\'an variando a lo largo del proceso, 
que podemos suponer que se desarrolla en instantes de tiempo sucesivos:
\begin{enumerate}
\item Una de todos los nodos a los que el l\'{\i}quido ha percolado hasta ese 
momento, a la que podemos llamar {\tt nodos\_visitados.}
\item Otra de los nodos alcanzados en el instante anterior, a la que podemos 
llamar {\tt nodos\_alcanzados.}
\end{enumerate}
\item Inicialmente, {\tt nodos\_visitados=nodos\_alcanzados=[(0,0)]}, pero, por 
ejemplo, puede ocurrir que en el siguiente instante tengamos {\tt 
nodos\_visitados=[(0,0),(1,0),(0,-1)]} y, por tanto, {\tt 
nodos\_alcanzados=[(1,0),(0,-1)].}
\item Entonces, la moneda se lanza, en cada instante, desde cada uno de los 
nodos alcanzados en el instante previo y, seg\'un el resultado, se actualiza la 
otra lista a\~nadi\'endole los nodos a los que el l\'{\i}quido ha percolado y 
que no estaban ya en ella.
\item  El proceso para {\itshape naturalmente} si en un instante no hay 
percolaci\'on desde ninguno de los nodos alcanzados, pero
para valores de $p$ altos debemos esperar que el l\'{\i}quido siga percolando 
indefinidamente y debemos parar el proceso nosotros.
\item  Podemos pararlo limitando el n\'umero veces que se va a repetir el 
intento de percolar, o bien parando el 
programa cuando se obtengan puntos con coordenadas mayores que un l\'{\i}mite 
prefijado. Es claro que no podemos determinar, usando el ordenador, si se va a 
producir {\itshape percolaci\'on hasta infinito} o no, pero variando estos 
l\'{\i}mites podemos hacernos una muy buena idea.
\item  Una variante del programa permitir\'{\i}a visualizar el resultado:
\begin{enumerate}
 \item Definimos una matriz $M$ , de tama\~no $(2N + 1)\times (2N + 1)$ con 
filas y columnas indexadas por enteros entre $0$ y
$2N$ (ambos inclusive) que va a consistir \'unicamente de ceros o unos.
 \item Cada vez que se {\itshape visita} un nodo $(n, m)$ hacemos que la 
entrada 
$M[n + N, m + N ]$  de la matriz sea un $1$, y los
nodos no visitados quedan representados por un cero en la matriz.

\item En este caso, para parar la percolaci\'on, no a\~nadimos a la lista de 
nodos alcanzados aquellos nodos que, aunque
el l\'{\i}quido hubiera percolado a ellos, est\'en fuera del cuadrado $[-N, N 
]\times [-N, N ].$

\item Podemos visualizar el resultado usando la instrucci\'on {\tt 
matrix\_plot(M).}
\end{enumerate}
\end{enumerate}
%\end{enumerate}
Una vez que hemos conseguido que el programa funcione, debemos estudiar, en 
funci\'on del valor de $0< p< 1$, cu\'ando se produce percolaci\'on hasta 
infinito. Es claro que si $p$ es peque\~na no la habr\'a  mientras que si es 
p\'oxima a $1$ deber\'{\i}a producirse, ya que, para cada nodo alcanzado 
sorteamos $4$ veces para ver si percola a alguno de los nodos vecinos. La 
probabilidad de que no percole a ninguno sabemos que vale $(1-p)^4$, y 
ser\'{\i}a muy peque\~na.

Debemos esperar entonces que en alg\'un valor $p_c$  de $p$ se produzca el 
cambio de comportamiento: para $p<p_c$ el l\'{\i}quido no percola hasta 
infinito mientras que para $p>p_c$ s\'{\i} lo hace. Decimos que $p_c$ es la 
{\itshape probabilidad cr\'{\i}tica}, y nos interesa, en primer lugar, 
determinarla experimentalmente.
\end{ejer}



\section{Ley de Benford}

La {\itshape Ley de Benford} se origin\'o como una observaci\'on emp\'{\i}rica, 
con el siguiente contenido:

{\itshape Decimos que una sucesi\'on cumple la Ley de Benford si la frecuencia 
del d\'{\i}gito $i$ como d\'{\i}gito dominante en los t\'erminos de la 
sucesi\'on es, con muy buena aproximaci\'on,  $B[i]:=log_{10}(1+\frac{1}{i+1})$.
}

Puedes leer un poco m\'as sobre la ley en 
\href{http://150.244.21.37/PDFs/MISCE/benford.pdf}{este art\'{\i}culo}.

\begin{ejer}

\begin{enumerate}
\item Define una función de {\sage}  $frecuencia\_fib(N)$ que tenga como 
argumento un entero $N$ y devuelva la lista de frecuencias de cada uno de los 
d\'{\i}gitos (1,2,3,4,5,6,7,8,9) como d\'{\i}gito dominante (i.e. el primero 
por la izquierda) de los n\'umeros en la sucesi\'on de Fibonacci que empezando 
en $1$  tiene  longitud $N$. Representa  gr\'aficamente las frecuencias 
obtenidas. 
\item Denotemos por $F$ la lista que devuelve $frecuencia\_fib(20000)$. 
Compara, usando una funci\'on de {\sage}, los valores $F[i]$ con los números  
$B[i]:=log_{10}(1+\frac{1}{i+1})$.  

\item Modifica la funci\'on del apartado $1)$ para calcular las mismas 
frecuencias pero para la sucesi\'on de potencias $2^j$ con $j$ recorriendo los 
enteros en el intervalo $[1,N]$. ?`Qu\'e observas al calcular las frecuencias 
con $N=20000$?
\item Modifica  la función del apartado $1)$ para calcular las mismas 
frecuencias pero para la sucesión de enteros en el intervalo $[1,N]$, y también 
para la sucesión de  cuadrados  $j^2$ con $j$ recorriendo los enteros en el 
intervalo $[1,N]$. ?`Qu\'e  observas al calcular las frecuencias con $N=20000$?
\item La {\itshape Ley de Benford} se cumple para muchas sucesiones generadas 
mediante una expresi\'on matem\'atica (no para todas), y para muchas sucesiones 
de datos num\'ericos obtenidos del mundo real, por ejemplo, mediante 
estad\'{\i}sticas. ?`Cu\'al puede ser la   diferencia fundamental entre 
sucesiones que cumplen bien la Ley de Benford y las que la cumplen medio bien o 
no la cumplen?

\end{enumerate}
\end{ejer}









%%\section{Dirichlet}

%%Mediante paseos aleatorios 
%%http://en.wikipedia.org/wiki/Discrete_Laplace_operator
%%http://www.math.uchicago.edu/~may/VIGRE/VIGRE2011/REUPapers/Guan.pdf
%%http://www.math.uchicago.edu/~lawler/reu.pdf pag23
\section{Tratamiento de im\'agenes}

Una aplicaci\'on interesante del \'Algebra Lineal consiste en usar la 
descomposici\'on en valores singulares (SVD) para la compresi\'on de 
im\'agenes. Intentamos reducir el tama\~no de un archivo que contiene una 
imagen 
sin una gran p\'erdida de calidad. 

La descomposici\'on en valores singulares de una matriz $m\times n$, con 
entradas reales,  $A$ consiste en la factorizaci\'on 
\[A=U\cdot \Lambda V^T,\]
\noindent con $U$ y $V$ matrices cuadradas ortogonales (i.e. con inversa igual 
a la transpuesta), y $\Lambda$ una matriz $m\times n$ con ceros fuera de la 
diagonal principal.  Los elementos no nulos de $\Lambda$ son los llamados 
{\itshape valores singulares} de $A$, y son las ra\'{\i}ces cuadradas positivas 
de los valores propios de la matriz sim\'etrica $A\cdot A^T$, que por ser 
sim\'etrica es diagonalizable. 

Puedes leer sobre la descomposici\'on y sus aplicaciones en
\href{http://150.244.21.37/PDFs/MISCE/SVD2.pdf}{este art\'{\i}culo}, y 
\href{http://150.244.21.37/PDFs/MISCE/SVD1.pdf}{en las notas} de Eugenio Hern\'andez sobre el 
tema.  


Adem\'as puedes consultar la
hoja de {\sage}
\href{http://sage.mat.uam.es:8888/home/pub/??/}{\tt 125-MISCE-imagenes.sws}, 
que contiene ejemplos en los que se aplica esta t\'ecnica. 

%%\section{FFT}
%%%???????????????????????????????????????????????????%%%%%%%%%%%%%%%



\section{Sistemas polinomiales de ecuaciones}
Como bien sabemos, los sistemas lineales de ecuaciones, con coeficientes en un 
cuerpo, se pueden resolver siempre sin salir del cuerpo de coeficientes. El 
m\'etodo est\'andar es la {\itshape reducci\'on gaussiana+sustituci\'on}, que 
se puede programar f\'acilmente y resuelve de manera exacta los sistemas 
lineales con coeficientes en los racionales o en un cuerpo finito\footnote{Es 
cierto que podemos encontrar las soluciones en un cuerpo finito de  un sistema 
de ecuaciones simplemente probando todas las posibles soluciones, pero si el 
cuerpo finito es grande, o el n\'umero de variables enorme, este m\'etodo de 
fuerza bruta no ser\'a el mejor posible.}.

Frecuentemente nos encontramos en la necesidad de resolver sistemas de 
ecuaciones no lineales, por ejemplo polinomiales o transcendentes (i.e. senos, 
cosenos, exponenciales, logaritmos, etc.) , y la triste realidad es que {\sc no 
existen m\'etodos generales de soluci\'on}. En cada caso tratamos de resolverlo 
usando alg\'un {\itshape truco}, que puede servir en ese ejemplo y quiz\'a en 
unos pocos similares. 

Sin embargo, en el caso de sistemas de ecuaciones polinomiales en varias 
variables es posible adaptar la reducci\'on gaussiana para aproximarnos de 
manera sistem\'atica a algo parecido a una soluci\'on. De hecho, el m\'etodo, 
que se conoce como  de {\itshape bases de Gr\"obner}, utiliza ideas que 
proceden del algoritmo de reducci\'on gaussiana junto con otras que proceden 
del algoritmo de Euclides para el m\'aximo com\'un divisor.  

Supongamos dado un sistema de ecuaciones polinomiales

%\begin{equation}\label{sis-eq}
 \begin{align*}\label{sis-eq}
  F_1(x_1,x_2,&\dots,x_m)=0\\
  F_2(x_1,x_2,&\dots,x_m)=0\\
  \vdots &  \\
  F_n(x_1,x_2,&\dots,x_m)=0
 \end{align*}
%\end{equation}

\noindent con los $F_i$ polinomios de grado $k_i$ en las $m$ variables. La 
teor\'{\i}a de  bases de Gr\"obner nos permite obtener un sistema de ecuaciones 
con las mismas soluciones que el dado, pero con una estructura {\itshape 
escalonada}. Concretamente, si suponemos que el sistema (\ref{sis-eq}) tiene un 
n\'umero finito de soluciones, obtenemos un sistema 

\begin{align*}%%\label{sis-eq-r}
  G_1(x_1,x_2,&\dots,x_m)=0\\
  G_2(x_1,x_2,&\dots,x_m)=0\\
  \vdots &  \\
  G_N(&x_m)=0
 \end{align*}

\noindent que tiene el mismo conjunto de soluciones que (\ref{sis-eq}), pero 
una ecuaci\'on, la \'ultima, con una s\'ola inc\'ognita. En las otras 
ecuaciones pueden no aparecer algunas de las variables, pero si {\sc 
supi\'eramos resolver} la ecuaci\'on polinomial con una sola inc\'ognita 
$G_N(x_m)=0$, podr\'{\i}amos sustituir sus soluciones en las otras ecuaciones y 
obtendr\'{\i}amos  sistemas de ecuaciones con $m-1$ inc\'ognitas. De esta 
manera 
podr\'{\i}amos, en principio, resolver el sistema original. 

El problema es que tampoco sabemos resolver ecuaciones polinomiales en una 
\'unica variable: sabemos que una ecuaci\'on polinomial en una variable de 
grado $k$ tiene $k$ ra\'{\i}ces complejas, contadas con su multiplicidad, pero 
\'unicamente podr\'{\i}amos, en general, calcular sus valores aproximados con 
un cierto n\'umero de d\'{\i}gitos prefijado. El problema es siempre la 
continuidad de los n\'umeros reales, necesitamos infinitos decimales para 
determinar un real no racional, frente a la finitud de la memoria RAM del 
ordenador. 

\subsection{Bases de Gr\"obner en \sage}
Usamos como cuerpo de coeficientes el de los n\'umeros racionales. 
\begin{enumerate}
 \item Comenzamos definiendo el anillo de polinomios en varias variables en el 
que vamos a operar:
 \begin{lstlisting}
  R = PolynomialRing(QQ,3,'xyz',order='lp')
 \end{lstlisting}

 \noindent $R$ es el anillo de polinomios con coeficientes racionales en tres 
variables $x,y,z.$ El \'ultimo par\'ametro indica c\'omo se ordenan los 
monomios, y como no entramos en los detalles de la teor\'{\i}a no nos interesa.
 
 
 \item Definimos el sistema de ecuaciones:
 \begin{lstlisting}
x,y,z = R.gens()
J = (2*x^2+y^2+z^2,x^2+2*y^2+z^2,x^2+y^2+2*z^2)*R
 \end{lstlisting}
 
\noindent que corresponde al sistema de ecuaciones
\begin{align*}\notag
  2x^2+y^2+z^2&=0,\\
  x^2+2y^2+z^2&=0,\\
  x^2+y^2+2z^2&=0.
 \end{align*}

Este sistema puede ser resuelto f\'acilmente {\itshape a mano}, pero en general 
los sistemas de ecuaciones cuadr\'aticas son dif\'{\i}ciles, y se puede 
demostrar que,  en un cierto sentido,  contienen toda la dificultad de los 
sistemas de ecuaciones de grados arbitrarios.

 \item C\'alculo de la base de Gr\"obner:
 
 \begin{lstlisting}
  B = J.groebner_basis();B
 \end{lstlisting}

 \begin{Output}
[x^2, y^2, z^2]
\end{Output}

Vemos que la \'unica soluci\'on del sistema es $(0,0,0)$ con una cierta 
multiplicidad que podr\'{\i}amos definir y calcular usando la base de Gr\"obner.

\item Otro ejemplo:

\begin{lstlisting}
x,y,z = R.gens()
J2 = (2*x^2+y^2+z^2+y*z,x^2+2*y^2+z^2+x*z,2*x^2+y^2+2*z^2+x*y)*R
B2 = J2.groebner_basis();B2
\end{lstlisting}
\noindent que corresponde al sistema de ecuaciones, algo m\'as complicado, 
\begin{align*}\notag
  2x^2+y^2+z^2+yz&=0,\\
  x^2+2y^2+z^2+xz&=0,\\
  x^2+y^2+2z^2+xy&=0.
 \end{align*}
 
 La base de Gr\"obner calculada es 
 \[\left[x^{2} + \frac{1}{2} y^{2} + \frac{1}{2} y z + \frac{1}{2} z^{2}, x
y -  y z + z^{2}, x z + \frac{3}{2} y^{2} - \frac{1}{2} y z +
\frac{1}{2} z^{2}, y^{3} - \frac{4}{5} z^{3}, y^{2} z - \frac{1}{10}
z^{3}, y z^{2} + \frac{1}{10} z^{3}, z^{4}\right]\]
\noindent y vemos que tiene una ecuaci\'on en una \'unica variable $z^4=0$, con 
soluci\'on $z=0$. En este caso es f\'acil, partiendo de que toda soluci\'on 
tiene la coordenada $z$ igual a cero, terminar de calcular todas las 
soluciones. 
 
\item Puede ocurrir que la base de Gr\"obner calculada no tenga ninguna 
ecuaci\'on en una sola variable. Se demuestra que esto \'unicamente puede 
ocurrir si el sistema original tiene infinitas soluciones complejas. Por 
ejemplo, si la base de Gr\"obner contiene un polinomio en dos variables 
$G_N(x_{m-1},x_m)=0$ basta ver que elegida  una de las infinitas soluciones 
complejas de esta ecuaci\'on en dos variables, que forman lo que llamamos 
{\itshape una curva algebraica plana con ecuaci\'on  $G_N(x_{m-1},x_m)=0$}, da 
lugar a una soluci\'on del sistema original en la que las \'ultimas dos 
coordenadas vienen dadas por las coordenadas de la soluci\'on elegida de la 
ecuaci\'on en dos variables. 

En este caso dir\'{\i}amos que la soluci\'on del sistema de ecuaciones original 
es una curva. 

\item La teor\'{\i}a de las bases de Gr\"obner no s\'olo permite aproximarnos a 
la resoluci\'on de sistemas de ecuaciones polinomiales de grados arbitrarios, 
sino que es la base para los c\'alculos efectivos, con ordenador, en 
Geometr\'{\i}a algebraica que es el estudio de conjuntos, curvas superficies, 
etc., dados como soluciones de sistemas de ecuaciones polinomiales. 
 
 
\end{enumerate}

\section{Ejercicios}

\begin{ejer}

En este ejercicio vamos a simular un acuario  de acuerdo a las siguientes 
reglas:
\begin{enumerate}
\item El acuario es un cubo $[0,N]\times[0,N]\times[0,N]$. Cada pez puede moverse en cada 
instante de tiempo incrementando, o decrementando,  una unidad una s\'ola de 
las coordenadas del punto,  con coordenadas enteras,  que ocupa. Si el punto 
tiene una de las coordenadas $0$ \'o $N$ el pez s\'olo puede moverse hacia el 
interior del acuario. Los movimientos de cada  pez son aleatorios y 
equiprobables entre todas las posibilidades de movimiento que tiene. En lugar 
de referirnos a puntos con coordenadas enteras los llamaremos {\itshape celdas} 
, y los pensamos como peque\~nos cubitos que dividen el acuario. Cada celda 
tiene $6$ celdas vecinas, que son a las que se pueden desplazar los peces.
\item Hay dos tipos de peces, $A$ y $B$ con $A$ un pez grande que  come  peces 
chicos de tipo  $B$. Los peces tienen sexo $M$ o bien $H$.
\item Los peces peque\~nos comen {\itshape placton}. El plancton no se mueve y 
cada $t_P$ períodos de tiempo coloniza cada una de las celdas vecinas con 
probabilidad $p_1$. Es decir, para cada una de las $6$ celdas vecinas pondremos 
una unidad de plancton si la moneda trucada ha salido $1$. Una celda puede 
tener varias unidades de plancton hasta llegar a un límite de saturación $S$.
\item Un pez peque\~no que llega a una celda que tiene plancton se queda en 
ella hasta que se acaba. En cada per\'{\i}odo de tiempo cada pez peque\~no 
come, si en la celda hay,  una unidad de plancton. Si en una celda que tiene 
plancton hay m\'as peces peque\~nos que plancton se sortea qu\'e peces comen. 
\item Un pez peque\~no que no ha comido durante $t_B$ unidades de tiempo muere 
(inanición). Un pez grande que no ha comido durante $t_A$ unidades de tiempo 
muere (inanición).  Un pez peque\~no vive durante $T_B$ unidades de tiempo 
(muerte natural) y uno grande durante $T_A$ (muerte natural).
\item Cuando un pez grande llega a una celda en la que hay peces peque\~os se 
puede comer uno con probabilidad $p_{A,B}$ y si no se lo come el pez peque\~no 
puede moverse, en el siguiente momento de tiempo,   a otra celda. Un pez grande 
que llega a una celda en la que hay peces peque\~nos se queda en ella hasta que 
ya no hay. En cada per\'{\i}odo de tiempo un pez grande s\'olo se puede comer a 
uno chico. Si en una celda hay m\'as peces grandes que peces peque\~nos se 
sortea cu\'ales de los grandes van a tener la posibilidad de comer. 
\item Cuando dos peces del mismo tipo y distinto sexo llegan a una misma celda 
producen descendencia de tamaños $D_A$ o $D_B$, la mitad de cada sexo. Si el 
pez está al borde de la inanición prefiere comer, si es posible, antes de 
reproducirse.
\item En el acuario hay inicialmente $n_A$ peces de tipo $A$, $n_B$ peces del 
tipo $B$ y $n_P$ unidades de plancton. La distribución inicial se puede hacer 
aleatoriamente y el número de peces de cada sexo debe ser igual.
\item Si en alg\'un momento del desarrollo del programa necesitas m\'as reglas 
puedes añadirlas, pero debes formularlas de la forma m\'as clara posible. El 
modelo se puede complicar todav\'{\i}a más, por ejemplo debe ser interesante 
permitir que los peces act\'uen de manera {\itshape inteligente} moviéndose no 
a una celda vecina aleatoriamente sino a la que contenga m\'as alimento (o 
mayor posibilidad de reproducirse).
\end{enumerate}
\end{ejer}

Una vez programado el modelo hay que estudiar el comportamiento a largo plazo 
dependiendo de los par\'ametros, y sobre todo buscar valores iniciales que no 
producen la extinci\'on.

\begin{ejer}
 En este ejercicio definimos un aut\'omata celular que
simula avalanchas, por ejemplo de nieve o de arena. Las reglas son como sigue:
  
  \begin{enumerate}%%[label=\textbf{\arabic*}]
  \item El estado del sistema en el momento $t,\ t=0,1,2,3,\dots,$ se representa
como una matriz $N\times N$, a la que llamamos,  por ejemplo,  $M(t)$. Los dos
\'{\i}ndices de la matriz var\'{\i}an entre $0$ y $N-1$,  con entradas
$M(t)_{ij}$ enteros no negativos. Esta matriz representa un {\itshape tablero} 
con
$N^2$ casillas, y el entero $M(t)_{ij}$ es el n\'umero de {\itshape granos de 
arena} en
la casilla de la fila $i$-\'esima y columna $j$-\'esima.
\item En este tablero cada casilla, por ejemplo $(i,j)$,  menos las del borde, 
tiene $4$ casillas vecinas: $(i+1,j),(i,j+1),(i-1,j),(i,j-1).$
\item Hemos definido un entero $K$ al que llamamos {\itshape valor 
cr\'{\i}tico}. Supondremos que $K\ge 4.$
\item Para \label{ev}pasar de $M(t)$ a $M(t+1)$:
Se revisa cada una de las casillas y si
$M(t)_{ij}>K$ hacemos $M(t+1)_{ij}=M(t)_{ij}-4$ e incrementamos en una unidad el
entero de cada uno de los vecinos de nuestra casilla. Hay que tener presente que
si la casilla es del borde tiene menos de $4$ vecinos, y hay que tratar ese caso
con cuidado. 
Si $M(t)_{ij}\le K$ hacemos $M(t+1)_{ij}=M(t)_{ij}.$
\item \label{ini}Generamos el estado inicial $M(0)$ considerando un tablero que
tiene en cada casilla un entero aleatorio del intervalo $[K+1,2K]$ y le
aplicamos el procedimiento del apartado anterior, de forma reiterada,  hasta que
la matriz ya no cambie. La matriz obtenida, que tiene en todas sus casillas un
entero menor o igual a $K$, ser\'a el estado inicial. 

\item \label{avalan}Para producir avalanchas primero generamos un estado inicial
$M(0)$ de la matriz,   elegimos una casilla al azar y le a\~nadimos un cierto
n\'umero de granos de arena, por ejemplo $I$, y, finalmente, iteramos  el
procedimiento del apartado {\ref{ev})} hasta que la matriz ya no var\'{\i}e. 

 \end{enumerate}
 
 
 
 \begin{enumerate}
  \item  {\sc Define} una funci\'on {\tt evoluci\'on(M,N,K)}, con $M$
una matriz $N\times N$ con entradas enteros no negativos y $K$ un entero, el
valor cr\'itico. La funci\'on debe implementar el procedimiento descrito en el
apartado  {\ref{ev}}, y, suponiendo que $M$ es el estado del tablero en el
momento $t$,  devolver el tablero en el $t+1.$ 
  
  \item  {\sc Define} una funci\'on {\tt ini(N,K)}, que implemente el
procedimiento descrito en el apartado {\ref{ini}}, y, por tanto, devuelva una
matriz que tiene en todas sus casillas un entero menor o igual a $K$. {\sc
Comprueba} que la matriz resultante de ejecutar {\tt ini(10,5)} tiene todas sus
entradas en el intervalo $[0,5].$

\item  {\sc Define} una funci\'on {\tt avalancha(N,K,I)}, que
implemente el procedimiento descrito en el apartado {\ref{avalan}} y devuelva el
estado final de la matriz y la diferencia entre el estado inicial y el final.
Esa diferencia nos permitir\'a visualizar el lugar en el que se ha producido la
avalancha y su intensidad. 

{\sc Trata} de obtener, variando los par\'ametros $N,K$ e $I$, avalanchas
{\itshape grandes}.
\end{enumerate}
\end{ejer}
